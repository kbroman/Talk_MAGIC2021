\documentclass[12pt,t,aspectratio=169]{beamer}
\usepackage{graphicx}
\setbeameroption{hide notes}
\setbeamertemplate{note page}[plain]
\usepackage{listings}

\input{header.tex}

%%%%%%%%%%%%%%%%%%%%%%%%%%%%%%%%%%%%%%%%%%%%%%%%%%%%%%%%%%%%%%%%%%%%%%
% end of header
%%%%%%%%%%%%%%%%%%%%%%%%%%%%%%%%%%%%%%%%%%%%%%%%%%%%%%%%%%%%%%%%%%%%%%

% title info
\title{QTL mapping in MAGIC \\ populations with R/qtl2}
\subtitle{Part 2}
\author{\href{https://kbroman.org}{Karl Broman}}
\institute{Biostatistics \& Medical Informatics, UW{\textendash}Madison}
\date{\href{https://kbroman.org}{\tt \scriptsize \color{foreground} kbroman.org}
\\[-4pt]
\href{https://github.com/kbroman}{\tt \scriptsize \color{foreground} github.com/kbroman}
\\[-4pt]
\href{https://twitter.com/kwbroman}{\tt \scriptsize \color{foreground} @kwbroman}
\\[2pt]
\scriptsize {\lolit Slides:} \href{https://kbroman.org/Talk_MAGIC2021}{\tt \scriptsize
  \color{foreground} kbroman.org/Talk\_MAGIC2021}
}


\begin{document}

% title slide
{
\setbeamertemplate{footline}{} % no page number here
\frame{
  \titlepage

  \vfill \hfill \includegraphics[height=6mm]{Figs/cc-zero.png} \vspace*{-3mm}

  \note{These are slides for talks I gave for Jeff Endelman's genetic mapping course
    at UW-Madison in Spring, 2021. They are based on a talk
    I gave at Michigan State University on 12 Dec 2019, which was a
    revised and expanded version of a talk I gave at the MAGIC workshop
    in Cambridge, UK, on 23 July 2019.


    Source: {\tt https://github.com/kbroman/Talk\_MAGIC2021} \\
    Slides: {\tt https://kbroman.org/Talk\_MAGIC2021}
}
} }



\begin{frame}[c]{MAGIC}

  \figw{Figs/ri8self.pdf}{1.0}

\end{frame}


\begin{frame}[c]{MAGIC is magic}

\vspace{-5mm}

\bbi
\item Genetic diversity

\item High-precision mapping

\item Predictable linkage disequilibrium

\item No rare alleles

\item Phenotype replicates to reduce individual variation

\item Pool phenotypes from multiple labs, environments, treatments

\item Genotype once

\item \hilit Cool name

\ei

\end{frame}




\begin{frame}{MAGIC lines}

 \vspace{5mm}

  \figw{Figs/valdar_genet2006.png}{0.9}

  \vspace{20mm}

\hfill {\footnotesize \href{http://www.genetics.org/content/172/3/1783.full}{\lolit Valdar et al., Genetics 172:1783, 2006}}

  \small {\hilit

  \vspace*{-24mm}
\hspace*{30mm} combine \hspace*{18mm} mix \hspace*{29mm} fix}

\vspace*{10pt}
\hspace*{8mm}
How many?

\vspace*{8pt}
\hspace*{8mm}
Which?


\vspace*{-35pt}
\hspace*{60mm}
How long?

\vspace*{-12pt}
\hspace*{100mm}
How?

\end{frame}




\begin{frame}{How many founders?}

\vspace{8mm}

  \begin{columns}

    \begin{column}{0.5\textwidth}
      {\hilit More}

{\small
\bi
\item More general use
\item More QTL
\item Greater precision
\item Estimate allele frequencies
\item Haplotype analysis in founders
  \ei
}

    \end{column}

    \begin{column}{0.5\textwidth}
      {\hilit Fewer}

{\small
\bi
\item Lower residual variance
\item Greater power for a particular QTL?
\item Better power for epistasis
\item Rare alleles are less rare
\ei
}
    \end{column}


  \end{columns}


\end{frame}



\begin{frame}[c]{Which founders?}

  \bbi
\item Diverse
\item Interesting
\item No breeding problems
\item Balanced: star phylogeny
  \ei

\end{frame}



\begin{frame}[c]{How much mixing?}

  \bbi
  \item More mixing $ \ \Rightarrow \ $ Greater mapping precision
\item ...but lower power for \emph{de novo\/} mapping
\item Potential for population structure, missing alleles
\item \hilit Random mating or curated mating?
\item \hilit Start with many random cross directions?
  \ei

\end{frame}


\begin{frame}[c]{Selfing or DH?}

\bbi
\item Inbreeding gives added recombination
\item But not so much as at the mixing stage
\item \hilit If doubled haploids are feasible, use them
  \ei

\end{frame}





\begin{frame}[c]{21 years of R/qtl}

\figw{Figs/rqtl_lines_code.pdf}{0.9}

\end{frame}




\begin{frame}{R/qtl cross types}

\bbi
\item backcross{\lolit , doubled haploids, haploid}

\item intercross

\item 2-way RIL {\lolit by selfing or sibling mating}

\item phase-known 4-way cross
\ei

\end{frame}






\begin{frame}[c]{}

\figh{Figs/rqtl2_3d.png}{0.9}


\end{frame}



\begin{frame}{R/qtl2 cross types}

\vspace*{10mm}

\bi
\item backcross, doubled haploids, haploid

\item intercross

\item 2-, 4-, 8-, 16-way RIL by selfing

\item 2-, 4-, 8-way RIL by sibling mating

\item 2-, 3-, 8-way advanced intercross

\item 6- and 19-way MAGIC

\item Diversity Outbred (DO) mice

\item F$_1$ of DO $\times$ inbred

\item general RIL or AIL
\ei

\end{frame}










\begin{frame}{Data files}

\includegraphics[width=0.8\textwidth]{Figs/phefile.pdf}

\only<2->{
  \vspace*{-50mm}
  \hspace*{15mm}
  \includegraphics[width=0.8\textwidth]{Figs/genfile.pdf}
}


\only<3->{
  \vspace*{-50mm}
  \hspace*{30mm}
  \includegraphics[width=0.8\textwidth]{Figs/fgfile.pdf}
}

\only<4->{
  \vspace*{-60mm}
  \hspace*{45mm}
  \includegraphics[width=0.4\textwidth]{Figs/pmapfile.pdf}
}


\end{frame}



\begin{frame}[c,fragile]{Control file (json or yaml)}
\begin{semiverbatim} \begin{lstlisting}[escapechar=!,]
{
  "description": "Arabidopsis MAGIC data, Gnan et al (2014)",
  !\color{foreground}{"crosstype": "magic19",}!
  "sep": ",",
  "na.strings": ["-", "NA"],
  "comment.char": "#",
  !\color{foreground}{"geno": "arabmagic_geno.csv",}!
  "founder_geno": "arabmagic_foundergeno.csv",
  "gmap": "arabmagic_pmap_tair9.csv",
  "pmap": "arabmagic_pmap_tair9.csv",
  "pheno": "arabmagic_pheno.csv",
  !\color{foreground}{"genotypes": {}!
  !\color{foreground}{  "A": 1}!
  !\color{foreground}{  "H": 2}!
  !\color{foreground}{  "B": 3}!
  !\color{foreground}{\},}!
  !\color{foreground}{"geno_transposed": true,}!
  "founder_geno_transposed": true
}
\end{lstlisting} \end{semiverbatim}
\end{frame}


\begin{frame}[c,fragile]{Control file (json or yaml)}
\addtocounter{framenumber}{-1}
\begin{semiverbatim} \begin{lstlisting}[escapechar=!]
{
  "description": "Arabidopsis MAGIC data, Gnan et al (2014)",
  !\color{vhilit}{"crosstype": "magic19",}!
  "sep": ",",
  "na.strings": ["-", "NA"],
  "comment.char": "#",
  !\color{foreground}{"geno": "arabmagic_geno.csv",}!
  "founder_geno": "arabmagic_foundergeno.csv",
  "gmap": "arabmagic_pmap_tair9.csv",
  "pmap": "arabmagic_pmap_tair9.csv",
  "pheno": "arabmagic_pheno.csv",
  !\color{foreground}{"genotypes": {}!
  !\color{foreground}{  "A": 1}!
  !\color{foreground}{  "H": 2}!
  !\color{foreground}{  "B": 3}!
  !\color{foreground}{\},}!
  !\color{foreground}{"geno_transposed": true,}!
  "founder_geno_transposed": true
}
\end{lstlisting} \end{semiverbatim}
\end{frame}



\begin{frame}[c,fragile]{Control file (json or yaml)}
\addtocounter{framenumber}{-1}
\begin{semiverbatim} \begin{lstlisting}[escapechar=!]
{
  "description": "Arabidopsis MAGIC data, Gnan et al (2014)",
  !\color{foreground}{"crosstype": "magic19",}!
  "sep": ",",
  "na.strings": ["-", "NA"],
  "comment.char": "#",
  !\color{vhilit}{"geno": "arabmagic_geno.csv",}!
  "founder_geno": "arabmagic_foundergeno.csv",
  "gmap": "arabmagic_pmap_tair9.csv",
  "pmap": "arabmagic_pmap_tair9.csv",
  "pheno": "arabmagic_pheno.csv",
  !\color{foreground}{"genotypes": {}!
  !\color{foreground}{  "A": 1}!
  !\color{foreground}{  "H": 2}!
  !\color{foreground}{  "B": 3}!
  !\color{foreground}{\},}!
  !\color{foreground}{"geno_transposed": true,}!
  "founder_geno_transposed": true
}
\end{lstlisting} \end{semiverbatim}
\end{frame}



\begin{frame}[c,fragile]{Control file (json or yaml)}
\addtocounter{framenumber}{-1}
\begin{semiverbatim} \begin{lstlisting}[escapechar=!]
{
  "description": "Arabidopsis MAGIC data, Gnan et al (2014)",
  !\color{foreground}{"crosstype": "magic19",}!
  "sep": ",",
  "na.strings": ["-", "NA"],
  "comment.char": "#",
  !\color{foreground}{"geno": "arabmagic_geno.csv",}!
  "founder_geno": "arabmagic_foundergeno.csv",
  "gmap": "arabmagic_pmap_tair9.csv",
  "pmap": "arabmagic_pmap_tair9.csv",
  "pheno": "arabmagic_pheno.csv",
  !\color{vhilit}{"genotypes": {}!
  !\color{vhilit}{  "A": 1}!
  !\color{vhilit}{  "H": 2}!
  !\color{vhilit}{  "B": 3}!
  !\color{vhilit}{\},}!
  !\color{foreground}{"geno_transposed": true,}!
  "founder_geno_transposed": true
}
\end{lstlisting} \end{semiverbatim}
\end{frame}



\begin{frame}[c,fragile]{Control file (json or yaml)}
\addtocounter{framenumber}{-1}
\begin{semiverbatim} \begin{lstlisting}[escapechar=!]
{
  "description": "Arabidopsis MAGIC data, Gnan et al (2014)",
  !\color{foreground}{"crosstype": "magic19",}!
  "sep": ",",
  "na.strings": ["-", "NA"],
  "comment.char": "#",
  !\color{foreground}{"geno": "arabmagic_geno.csv",}!
  "founder_geno": "arabmagic_foundergeno.csv",
  "gmap": "arabmagic_pmap_tair9.csv",
  "pmap": "arabmagic_pmap_tair9.csv",
  "pheno": "arabmagic_pheno.csv",
  !\color{foreground}{"genotypes": {}!
  !\color{foreground}{  "A": 1}!
  !\color{foreground}{  "H": 2}!
  !\color{foreground}{  "B": 3}!
  !\color{foreground}{\},}!
  !\color{vhilit}{"geno_transposed": true,}!
  "founder_geno_transposed": true
}
\end{lstlisting} \end{semiverbatim}
\end{frame}




\begin{frame}[fragile,c]{Reading data into R}


\begin{center} \begin{minipage}[c]{11.7cm} \begin{semiverbatim}
\lstset{basicstyle=\large}
\begin{lstlisting}[linewidth=11.7cm]
library(qtl2)
arab <- read_cross2("arab_magic.json")
\end{lstlisting}
\end{semiverbatim} \end{minipage} \end{center}


\onslide<2>{
\vfill
\hfill \begin{minipage}[c]{6cm}

  \small
  {\color{title} 19-way Arabidopsis MAGIC} \\
  Kover et al. (2009) PLoS Genet \\
  Gnan et al. (2014) Genetics \\
  {\tt github.com/rqtl/qtl2data}

\end{minipage}
}

\end{frame}


\begin{frame}[c]{Data diagnostics}


\hfill \begin{minipage}{11.7cm}
{\lolit See} Broman et al. (2019) Cleaning genotype data from \\
{\color{background} See} Diversity Outbred mice. G3 9:1571--1579 \\[12pt]
{\color{background} See See} \href{https://doi.org/10.1534/g3.119.400165}{doi: 10.1534/g3.119.400165}
\end{minipage}

\end{frame}



\begin{frame}[c]{Genotype reconstruction}


\only<1>{\figw{Figs/geno_reconstruct.pdf}{0.95}}
\only<2>{\figw{Figs/geno_reconstruct_B.pdf}{0.95}}

\end{frame}




\begin{frame}[c,fragile]{Genotype reconstruction}

\begin{center} \begin{minipage}[c]{11.5cm} \begin{semiverbatim}
\begin{lstlisting}[linewidth=11.5cm]
gmap <- insert_pseudomarkers(arab$gmap, step=0.2, stepwidth="max")
pmap <- interp_map(gmap, arab$gmap, arab$pmap)

pr <- calc_genoprob(arab, gmap, error_prob=0.002, cores=24)
\end{lstlisting}
\end{semiverbatim} \end{minipage} \end{center}

\end{frame}



\begin{frame}[c]{Genome scan}

\only<1>{\figw{Figs/scan_hk.pdf}{1.0}}
\only<2>{\figw{Figs/scan_lmm.pdf}{1.0}}
\only<3>{\figw{Figs/scan_loco.pdf}{1.0}}

\end{frame}

\begin{frame}[c]{Genome scan}

\figw{Figs/scan_seedwt.pdf}{1.0}

\end{frame}


\begin{frame}[c,fragile]{Genome scan}

\begin{center} \begin{minipage}[c]{11.3cm} \begin{semiverbatim}
\begin{lstlisting}[linewidth=11.3cm]
out_hk <- scan1(pr, arab$pheno, cores=24)

operm_hk <- scan1perm(pr, arab$pheno, n_perm=1000, cores=24)

k <- calc_kinship(pr, cores=24)
out_lmm <- scan1(pr, arab$pheno, k, cores=24)

k_loco <- calc_kinship(pr, "loco", cores=24)
out_loco <- scan1(pr, arab$pheno, k_loco, cores=24)
\end{lstlisting}
\end{semiverbatim} \end{minipage} \end{center}

\end{frame}



\begin{frame}[c]{SNP association scan}

\only<1>{\figw{Figs/snp_asso.pdf}{1.0}}
\only<2>{\figw{Figs/snp_asso_B.pdf}{1.0}}
\only<3>{\figw{Figs/snp_asso_B_logp.pdf}{1.0}}

\end{frame}



\begin{frame}[c]{SNP association scan}

\only<1>{\figw{Figs/snp_asso_C.pdf}{1.0}}
\only<2>{\figw{Figs/snp_asso_C_logp.pdf}{1.0}}

\end{frame}


\begin{frame}[c,fragile]{SNP association scan}

\begin{center} \begin{minipage}[c]{11.3cm} \begin{semiverbatim}
\begin{lstlisting}[linewidth=11.3cm]
snp_pr <- genoprob_to_snpprob(pr, arab)

out_snps <- scan1(snp_pr, arab$fruit, cores=24)
\end{lstlisting}
\end{semiverbatim} \end{minipage} \end{center}

\end{frame}





\begin{frame}[c]{QTL effects}

\only<1>{\figw{Figs/coef_fl.pdf}{1.0}}
\only<2>{\figw{Figs/blup_fl.pdf}{1.0}}

\end{frame}



\begin{frame}[c]{QTL effects}

\figw{Figs/blup_sw.pdf}{1.0}

\end{frame}


\begin{frame}[c,fragile]{QTL effects}

\begin{center} \begin{minipage}[c]{11.3cm} \begin{semiverbatim}
\begin{lstlisting}[linewidth=11.3cm]
fl_peak <- max(out_hk, pmap, lodcolumn="fruit_length")
fl_pr <- pull_genoprobpos(pr, pmap, fl_peak$chr, fl_peak$pos)

fl_fit1 <- fit1(fl_pr, arab$pheno[,"fruit_length"])
fl_blup <- fit1(fl_pr, arab$pheno[,"fruit_length"], blup=TRUE)
\end{lstlisting}
\end{semiverbatim} \end{minipage} \end{center}

\end{frame}









\begin{frame}{Goals}

  \bbi
\item Genotype reconstructions from external software

\item Sequencing-based genotype data

\item Multiple-QTL models

\item QTL $\times$ environment interactions

\item Interactive data visualization
  \ei

\end{frame}




\begin{frame}[c]{}

\Large

Slides: \href{https://kbroman.org/Talk_MAGIC2021}{\tt
  kbroman.org/Talk\_MAGIC2021} \quad
\includegraphics[height=5mm]{Figs/cc-zero.png}

\vspace{7mm}

\href{https://kbroman.org}{\tt \lolit kbroman.org}

\vspace{7mm}

\href{https://kbroman.org/qtl2}{\tt kbroman.org/qtl2}

\vspace{7mm}

\href{https://github.com/kbroman}{\tt \lolit github.com/kbroman}

\vspace{7mm}

\href{https://twitter.com/kwbroman}{\tt \lolit @kwbroman}


\end{frame}

\end{document}
